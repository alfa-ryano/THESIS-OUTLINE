\documentclass[12pt, a4paper]{report} \usepackage[titletoc]{appendix}
%\linespread{1.5}
%\usepackage{lineno}
%\linenumbers
\usepackage{makecell}
\usepackage{kantlipsum}
\usepackage{enumitem}
\usepackage{tabularx}
\usepackage{appendix}
\usepackage{multirow}
\usepackage{hhline}
\usepackage{array}
\usepackage{caption}
\usepackage{subcaption}
\usepackage{float}
\usepackage{graphicx}
\graphicspath{{images/}} 
\usepackage{geometry}
\geometry{a4paper,left=3cm,top=3cm,bottom=3cm,right=3cm}
\usepackage{array}
\usepackage{multirow}
\usepackage{hyperref}
\hypersetup{colorlinks=true,allcolors=blue}
\usepackage{hypcap}
\usepackage[linesnumbered,ruled]{algorithm2e}
\usepackage{courier}
\usepackage{listings}
\lstset{
basicstyle=\ttfamily,
frame=none, 
breaklines=true,
numbers=left,
xleftmargin=2.5em,
framexleftmargin=0em,
emphstyle=\textbf,
float=t
}
\lstdefinestyle{ocl}{
emph={
context, inv
}
}
\lstdefinestyle{change-based persistence}{
basicstyle=\ttfamily\scriptsize,
emph={
session, create, of, type,
set, to, add, hire
}
}
\lstdefinestyle{xmi}{
basicstyle=\ttfamily\scriptsize,
emph={
Node, children,
Employee, manages
}
}
\lstdefinestyle{xml}{
basicstyle=\ttfamily\scriptsize,
emph={
register, create, add, to, resource,
from, eattribute, remove, ereference,
set, unset, session, Roy, Jen,
Moss, Richmond
}
}
\lstdefinestyle{java}{
basicstyle=\ttfamily\scriptsize,
emph={
case, UNSET,
instanceof, else, if, void,
new, UnsetEAttributeEvent,
UnsetEReferenceEvent,
@override, public, class, extends
}
}
\lstdefinestyle{eol}{
basicstyle=\ttfamily\scriptsize,
emph={
var, new, for, in, create, set, of, with, 
unset, to, add, remove, delete, register,
from, position, from, move-within, session, \.
}
}

\setlength{\parindent}{1cm}
\setlength{\parskip}{0.1cm}


\begin{document}

\begin{titlepage}
\begin{center}

\textbf{Progress Report}
\vspace{1cm}

\textbf{\large Change-Based Model Persistence}
\vspace{1cm}

Alfa Ryano Yohannis\\
ary506@york.ac.uk
\vspace{1cm}

Supervisors:\\
Dimitris Kolovos\\
Fiona Polack\\
\vspace{1cm}

Department of Computer Science\\
University of York\\
United Kingdom\\
\vspace{1cm}
\today

\vfill

\end{center}
\end{titlepage}


\begin{abstract}
\addcontentsline{toc}{chapter}{Abstract}
Most of the models in Model-Driven Engineering are persisted in state-based formats. However, this type of persistence is problematic when it comes to detecting changes in large-scale models. As an alternative, this work proposes a change-based approach that involves persisting the full sequence of changes made to models. Persisting models in a change-based format has the potential to deliver benefits over state-based persistence, such as the ability to detect changes much faster and more precisely, which can then yield positive knock-on effects on helping developers compare and merge models in collaborative modelling environments. Nevertheless, change-based persistence also comes with downsides, such as ever-growing model file sizes and increased model loading time. So far, an initial change-based persistence format for EMF models has been developed, and an algorithm to reduce the loading time of change-based models has been proposed. Based on this work's interim results, the change-based approach persists changes in models faster than its state-based counterpart, and the proposed algorithm has successfully loaded change-based models faster than loading the models naively. The initial implementation has been presented in a workshop, and the proposed algorithm has been submitted to a conference and is currently is under review. A research plan to complete this work in the next two years is also explained in this report.
\end{abstract}

\tableofcontents
\addcontentsline{toc}{chapter}{Contents}

\listoffigures
\newpage

\listoftables
\newpage

\lstlistoflistings
\newpage

\chapter{Introduction}
\label{ch:introduction}
This Chapter briefly presents the background of this work as well as the research questions that will be addressed in this project. Several research objectives are then defined to answer the research questions. Lastly, research outputs and scoping are also presented. 

\section{Background}
\label{sec:background}
Most of the models in the context of Model-Driven Engineering are persisted in state-based formats. In such approaches, model files contain snapshots of the models' contents, and activities like version control and change detection are left to external systems such as file-based version-control systems and model differencing facilities. Activities such as change-detection -- identifying parts that have changed in a model compared to a previous version/ancestor -- and model comparison -- finding differences between models -- are computationally consuming for state-based models.

As an alternative to state-based persistence, this work proposes that a model can also be persisted in a change-based format, which persists the full sequence of \emph{changes} made to the model instead. The concept of change-based persistence is not new and has been used in persisting changes to software, object-oriented databases, and hierarchical documents \cite{DBLP:journals/entcs/RobbesL07,DBLP:conf/sde/LippeO92,DBLP:conf/caise/IgnatN05}. The change-based approach can improve detecting differences more precisely at the semantic level -- that is by providing finer-granularity information (e.g. types of changes, the order of the changes, elements that were changed, previous values, etc.) -- and therefore provide support to resolve them \cite{mens2002state}. The ordered nature of change-based persistence means that changes made to a model can be identified sequentially without having to explore and compare all elements of the model and its previous version. Based on these arguments, this work proposes change-based persistence as an alternative approach to state-based persistence for models conforming to 3-layer metamodelling architectures such as EMF and MOF. Persisting models in change-based format will bring a number of envisioned benefits over state-based persistence, such as the ability to detect changes much faster and more precisely, which can then have positive knock-on effects on helping developers compare and merge models in collaborative modelling environments. 

Nevertheless, change-based persistence also comes with downsides, such as ever-growing model files and increased model loading time. A model that is modified frequently will increase considerably in file size since every change is added to the file. The increased file size (proportional to the number of persisted changes) will increase the loading time of the model since all changes have to be replayed to reconstruct the model's eventual state. These downsides have to be mitigated to enable the practical adoption of change-based persistence. One approach to reducing the file size of change-based models is by removing changes that do not affect the eventual state of the model. For the increased loading time, it can be mitigated by ignoring -- i.e. not replaying -- changes that are cancelled out by later changes or employing change-based and state-based persistence side-by-side so that the benefits of state-based persistence on loading time can be obtained.   

\section{Research Questions}
\label{sec:research_questions}
The hypothesis of this work is that, \textbf{``Change-based persistence reduces the execution time of model change-detection, model comparison, and model merging for large models compared to their execution time in state-based persistence, with acceptable trade-offs on loading and persisting time, memory footprint, and storage space consumption''}. The execution time is the time required to complete the processes (e.g. change-detection, model comparison, model merging, or persisting changes). Model change-detection is identifying changed elements of a model compared to its previous version/ancestor while model comparison is finding the differences between two models that come from the same ancestor. Model merging is reconciling two models that come from the same ancestor and merge them to produce a new model. Using the term "large models" we refer to models with more than 1M elements, consistently with \cite{daniel2016neoemf,pagan2011morsa}. Model load time is the amount of time required to load a model into memory. Persisting changes is saving changes made to a model into a persistent representation (e.g. a file). For state-based models, this requires saving the entire model. Memory footprints are the sizes of memory used to execute the processes. Disk space consumption is the amount of storage consumed by the persistence.  

To assess the validity of the hypothesis, this work aims to answer the following research questions: 
\begin{enumerate} 
\item \textbf{How to persist models in a change-based format? How does it perform compared to state-based persistence on saving changes?} 

The concept of change-based persistence has to be translated into an implementation in a modelling framework context so that it can be applied for model persistence, and therefore its impact on model change-detection, model comparison, and model merging can be assessed. It is expected that every change made to a model can be persisted by the implementation. Replaying all the persisted changes can reconstruct the same model as the model persisted in state-based format. It is also expected that change-based persistence will outperform state-based persistence on time required for saving changes since change-based persistence will only require persisting changes of a model while state-based persistence will persist the whole model. 

\item \textbf{How to mitigate the ever-growing file size and increased loading time of change-based models? To what extent can they be reduced?} 

Change-based persistence comes with the downsides of larger file sizes and increased loading time. Mitigating these side effects will is essential for facilitating the practical adoption of change-based persistence. For the increased file size, the size can be reduced by removing changes that do not affect the eventual state of the model. For the increased loading time, it can be mitigated by ignoring -- not replaying -- changes that are cancelled out by subsequent changes or employing change-based and state-based persistence side-by-side (hybrid approach). Employing both persistence side-by-side can maintain the benefit of state-based persistence on loading time. It is expected that the mitigation approaches will (1) lead to loading time that is closer to the loading time of state-based format, (2) significantly load models faster than loading change-based models naively, and (3) significantly reduce change-based model file sizes compared to the naive approach on saving change-based model files. 

For reducing the loading time using change-based and state-based persistence side-by-side, the impact of the approach will also be investigated on several qualities. It is expected that a hybrid persistence approach will: (1) be significantly slower than change-based persistence on persisting changes and slightly slower than state-based persistence on saving time as the hybrid approach will need to save changes to the other two types of persistence, (2) consume more disk space compared to change-based or state-based persistence since the hybrid approach will use them both simultaneously, and (3) facilitate loading time that is closer to the loading time of state-based persistence. 

\item \textbf{How to detect changes in change-based models -- comparing them to their ancestors/previous versions? To what extent does change-detection in change-based models perform compared to change-detection in state-based models?} 

The purpose of using change-based persistence in this work is to improve change-detection. The change-based persistence will have change-detection time that is smaller than the change-detection time of state-based persistence.        

\item \textbf{How to compare change-based models that come from the same ancestor? How does the comparison of change-based models perform compared to the state-based model comparison?} 

The knock-on effect of faster change-detection on model comparison will also be investigated. Due to the nature of change-based models, the mechanism to perform change-based model comparison will differ substantially from the current state-based model comparison. It is expected that comparison of change-based models will be significantly faster than the comparison of state-based models.   

\item \textbf{How to merge different change-based models that come from the same ancestor? How does the merging perform compared to model merging in state-based persistence?}

Another knock-on effect of faster change-detection of change-based persistence is faster model merging. Similar to the change-based model comparison, the mechanism to merge change-based models will differ substantially from merging state-based models. It is expected that the change-based model merging will be much faster than state-based model merging.   

\end{enumerate}

\section{Research Objectives}
\label{sec:research_objectives}
This research aims to meet the following research objectives to answer the research questions.
\begin{enumerate}
\item Develop an implementation of change-based persistence so it can be applied to persist models in change-based format, and evaluate the correctness of change-based models that it produces and its performance on saving changes against state-based persistence. 
\item Propose approaches to reduce the increased file size and loading time of change-based persistence models, and evaluate their performance against naive approaches. The increased file size will be reduced by removing changes that do not affect the eventual state of the model. Meanwhile, the increased loading time will be reduced by ignoring -- not replaying -- changes that are cancelled out by preceding changes or employing change-based and state-based persistence side-by-side. Employing both persistence side-by-side can maintain the benefit of state-based persistence on loading time. 
\item Develop a solution to detect changes in change-based models, and compare its execution time and memory footprint against change-detection in state-based models.
\item Develop a solution to compare change-based models, and compare its execution time and memory footprint against model comparison in state-based models.
\item Develop a solution to merge different change-based models, and compare its execution time and memory footprint against model-merging in state-based models. 
\end{enumerate}

\section{Research Outputs}
\label{sec:research_outputs}
By the end of this research, these following outputs will have been produced:
\begin{enumerate}
\item Prototypes for change-based persistence and hybrid model persistence. 
\item Solutions -- including their implementation and evaluation -- for file size and loading time reduction, change-detection (finding parts that already changed of a model compared to its previous version/ancestor), model comparison (finding differences between models that come from the same ancestor), and model merging of change-based persistence.
\item A publication for each research question, and a thesis documenting the outcomes of this research.
\end{enumerate}


\section{Research Scope}
\label{sec:research_scope}
All prototypes in this research will be developed on top of Eclipse Modelling Framework. Since this research will also use change-based and state-based persistence side-by-side, an existing instance of state-based persistence is required for executing the implementation and evaluation. NeoEMF \cite{daniel2016neoemf}, a recent work that leverages the use of NoSQL databases for large-scale model persistence, is considered pertinent for this research. 

\chapter{Thesis Structure}
\label{sec:Thesis Structure}

- Description of the purpose and contents of the chapter
- Why this is the appropriate place for this chapter?
- What research has been completed for this chapter?
- What has been written for this chapter?
- What material has been included in a publication? Or Publication related to this chapter?
- How to complete the outstanding work: research, review, and writing?

\section{Chapter 1: Introduction}
\label{sec:chapter_1_introduction_plan}
\begin{enumerate}
\setlength\itemsep{0pt}
\item Background.
\item Research Questions.
\item  Research Objectives.
\item  Research Outputs.
\item  Research Scope.
\end{enumerate}

\section{Chapter 2: Literature Review}
\label{sec:chapter_2_literature_review_plan}

\begin{enumerate}
\setlength\itemsep{0pt}
\item Change-based approach in software engineering.
\item Change-based model persistence (including benefits and drawbacks).
\item State-based model persistence (including benefits and drawbacks).
\item Text comparison and merging.
\item State-based model comparison and merging.
\item Change-based model comparison and merging.
\end{enumerate}

\section{Chapter 3: Change-based Model Persistence}
\label{sec:chapter_3_Change-based_model_ersistence_plan}

\begin{enumerate}
\setlength\itemsep{0pt}
\item Concept of Change-based Model Persistence.
\item Example.
\item Implementation.
\item Implementation example.
\end{enumerate}

\section{Chapter 4: Optimised Loading of Change-based Model Persistence}
\label{sec:chapter_4_optimised_loading_change_based_model_persistence}

\begin{enumerate}
\setlength\itemsep{0pt}
\item Introduction.
\item The Optimisation Approach.
\item Evaluation and Results.
\item Discussion.
\item Conclusions.
\end{enumerate}

\section{Chapter 5: Hybrid Model Persistence}
\label{sec:chapter_5_hybrid_model_persistence}

\begin{enumerate}
\setlength\itemsep{0pt}
\item Introduction.
\item The Hybrid Model Persistence Approach.
\item Evaluation and Results.
\item Discussion.
\item Conclusions.
\end{enumerate}

\section{Chapter 6: Change-based Model Comparison}
\label{sec:chapter_6_change_based_model_comparison}

\begin{enumerate}
\setlength\itemsep{0pt}
\item Introduction.
\item Comparison between two sequential versions of a model.
\item Comparison between two parallel versions of a model. 
\item Evaluation and results.
\item Discussion.
\item Conclusions.
\end{enumerate}

\section{Chapter 7: Change-based Model Merging}
\label{sec:chapter_7_change_based_model_Merging}

\begin{enumerate}
\setlength\itemsep{0pt}
\item Introduction.
\item Merging between two sequential versions of a model.
\item Merging between two parallel versions of a model. 
\item Detecting conflicts and dependency between events.
\item One-way (all-right-to-left or all-left-to-right) and two-way merging.
\item Automatic and user-defined merging strategies. 
\item Evaluation and results.
\item Discussion.
\item Conclusions.
\end{enumerate}

\section{Chapter 8: Change-based Model Compression}
\label{sec:chapter_8_change_based_model_compression}

\begin{enumerate}
    \setlength\itemsep{0pt}
    \item Introduction.
    \item The Compression Approach.
    \item Evaluation and Results.
    \item Discussion.
    \item Conclusions.
\end{enumerate}

\section{Chapter 9: Conclusions and Future Work}
\label{sec:chapter_9_conclusions}

\begin{enumerate}
    \setlength\itemsep{0pt}
\item Conclusions from every chapter.
\item Future work.
\end{enumerate}

\chapter{Plan}
\label{sec:plan}

\section{Chapter 1: Introduction}
\label{sec:chapter_1_introduction}

\begin{table}[h]
\centering
\caption{The Time Table for the Chapter 1: Introduction.}
\label{table:chapter_1_introduction_time}
\begin{tabular}{ c c c }
    \hline 
    \textbf{\thead{Task}} & \textbf{\thead{Completed}} & \textbf{\thead{Time}} \\
    \hline 
    \makecell[l]{\textbf{Research}} & & \\
    \makecell[l]{\quad Background} & Yes &  \\
    \makecell[l]{\quad Research Questions} & Yes &  \\
    \makecell[l]{\quad Research Objectives} & Yes &  \\
    \makecell[l]{\quad Research Outputs} & Yes &  \\
    \makecell[l]{\quad Research Scope} & Yes &  \\
    \hline
    \makecell[l]{\textbf{Publication}} &  &  \\
    \hline
    \makecell[l]{\textbf{Writing-up}} &  &   \\
    \makecell[l]{\quad Background} & No & 1/5 day \\
    \makecell[l]{\quad Research Questions} & No & 1/5 day \\
    \makecell[l]{\quad Research Objectives} & No & 1/5 day \\
    \makecell[l]{\quad Research Outputs} & No & 1/5 day \\
    \makecell[l]{\quad Research Scope} & No & 1/5 day \\
    \hline
\end{tabular}
\end{table}

\section{Chapter 2: Literature Review}
\label{sec:chapter_2_literature_review}
\begin{table}[h]
\centering
\caption{The Time Table for the Chapter 2: Literature Review.}
\label{table:chapter_2_literature_review_time}
\begin{tabular}{ c c c }
    \hline 
    \textbf{\thead{Task}} & \textbf{\thead{Completed}} & \textbf{\thead{Time}} \\
    \hline 
    \makecell[l]{\textbf{Research}} & & \\
    \makecell[l]{\quad Background} & Yes &  \\
    \makecell[l]{\quad Research Questions} & Yes &  \\
    \makecell[l]{\quad Research Objectives} & Yes &  \\
    \makecell[l]{\quad Research Outputs} & Yes &  \\
    \makecell[l]{\quad Research Scope} & Yes &  \\
    \hline
    \makecell[l]{\textbf{Publication}} &  &  \\
    \hline
    \makecell[l]{\textbf{Writing-up}} &  &   \\
    \makecell[l]{\quad Background} & No & 1/5 day \\
    \makecell[l]{\quad Research Questions} & No & 1/5 day \\
    \makecell[l]{\quad Research Objectives} & No & 1/5 day \\
    \makecell[l]{\quad Research Outputs} & No & 1/5 day \\
    \makecell[l]{\quad Research Scope} & No & 1/5 day \\
    \hline
\end{tabular}
\end{table}


\section{Chapter 3: Change-based Persistence}
\label{sec:chapter_3_Change-based Persistence}

\section{Chapter 4: Optimised Loading of Change-based Persistence}
\label{sec:chapter_4_optimised_loading_plan}

\section{Chapter 5: Hybrid Model Persistence}
\label{sec:chapter_5_hybrid_model_persistence_plan}

\section{Chapter 6: Change-based Model Comparison}
\label{sec:chapter_6_change_based_model_comparison_plan}

\section{Chapter 7: Change-based Model Merging}
\label{sec:chapter_7_change_based_model_merging_plan}

\section{Chapter 8: Change-based Model Compression}
\label{sec:chapter_8_change_based_model_compression_plan}

\section{Chapter 9: Conclusions and Future Work}
\label{sec:chapter_9_conclusions_plan}


%For each proposed chapter of your thesis, give a bullet summary of outstanding tasks for (a) research and (b) writing up; for each bullet state the estimated time required.
%Provide a diagrammatic timetable that shows each task, and any publication proposals.  
%Your timetable should match your time estimates
%If you cannot create a reasonable plan for completion within the normal 3-year (fulltime) enrolment, and your work cannot be scoped down to fit, you need to be very clear why you need to use time in the permitted overrun year.


\chapter{Publications}
\label{ch:publications}
Three papers have been written. The first paper \cite{yohannis2017turning} has been presented in the FlexMDE 2017 workshop, the second paper has been accepted and will be presented in the ECMFA 2018 conference, and the third one has been submitted to ASE 2018 and currently under review.
\begin{enumerate}
\item A. Yohannis, F. Polack, and D. Kolovos, ``Turning Models iInside Out," in Proceedings of the 3rd Workshop on Flexible Model Driven Engineering co-located with ACM IEEE 20th International Conference on Model Driven Engineering Languages and Systems (MoDELS 2017), 2017.
\item  A. Yohannis, H. Hoyos Rodriguez, F. Polack, and D. Kolovos, ``Towards Efficient Loading of Change-based Models," in Proceedings of the 14th European Conference on Modelling Foundations and Applications (ECMFA 2018) co-located with Software Technologies: Applications and Foundations (STAF 2018), 2018.
\item  A. Yohannis, H. Hoyos Rodriguez, F. Polack, and D. Kolovos, ``Towards Hybrid Model Persistence," submitted to the 33rd IEEE/ACM International Conference on Automated Software Engineering (ASE 2018), 2018 (under review).
\end{enumerate}

\bibliographystyle{IEEEtran}
\bibliography{references}

%\begin{appendices}
%\end{appendices}

\end{document}